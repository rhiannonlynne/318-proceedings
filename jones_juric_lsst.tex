% iaus2esa.tex -- sample pages for Proceedings IAU Symposium document class
% (based on v1.0 cca2esam.tex)
% v1.04 released 17 May 2004 by TechBooks
%% small changes and additions made by KAvdH/IAU 4 June 2004
% Copyright (2004) International Astronomical Union

\NeedsTeXFormat{LaTeX2e}

\documentclass{iau}
\usepackage{graphicx}

\title[Asteroids in LSST] %% give here short title %%
{Asteroid Discovery and Characterization with the Large Synoptic Survey Telescope (LSST)}

\author[R. Lynne Jones \& Mario Juric]   %% give here short author list %%
{R. Lynne Jones$^1$
%%  \thanks{Present address: Department of Astronomy, University of
%%  Washington, Seattle, USA},
 \and Mario Juric$^2$}

\affiliation{$^1$University of Washington\\ email: {\tt lynnej@uw.edu} \\[\affilskip]
$^2$University of Washington \\email: {\tt mjuric@astro.astro.washington.edu}}

\pubyear{2015}
\volume{318}  %% insert here IAU Symposium No.
\setcounter{page}{1}
\jname{Asteroids: New Observations, New Models}
\editors{Steve Chesley, Alessandro Morbidelli, Robert Jedicke \&
  Davide Farnocchia eds.}
\begin{document}

\maketitle

\begin{abstract}
The Large Synoptic Survey Telescope (LSST) will be a ground-based,
optical, all-sky, rapid cadence survey project wiht tremendous
potential for discovering and characterizing asteroids.

With LSST's large 6.5m diameter primary mirror, a wide 9.6 square
degree field of view 3.2 Gigapixel camera, and rapid observational
cadence, LSST will discover more than 5 million asteroids over its ten
year survey lifetime. With a single visit limiting magnitude of 24.5
in $r$ band, LSST will be able to detect asteroids in the Main Belt
down to sub-kilometer sizes.  The current strawman for the LSST survey
strategy is to obtain two visits (each `visit' being a pair of
back-to-back 15s exposures) per field, separated by about 30 minutes,
covering the entire visible sky every 3-4 days throughout the
observing season, for ten years.

The catalogs generated by LSST will increase the known number of small
bodies in the Solar System by a factor of 10-100 times, among all
populations. The median number of observations for Main Belt asteroids
will be on the order of 200-300, with Near Earth Objects receiving a
median of 90 observations. These observations will be spread among
$ugrizy$ bandpasses, providing photometric colors and the possibility
of performing sparse lightcurve inversion to determine rotation
periods, spin axes, and shape information.

These catalogs will be created using automated detection software, the
LSST Moving Object Processing System (MOPS), that will take advantage
of the carefully characterized LSST optical system, cosmetically
clean camera, and recent improvements in difference imaging. [short
description of the DM section here].
\keywords{Keyword1, keyword2, keyword3, etc.}
%% add here a maximum of 10 keywords, to be taken form the file <Keywords.txt>
\end{abstract}

\firstsection % if your document starts with a section,
                     % remove some space above using this command.
\section{Introduction}

Why we build LSST, what is LSST. Why LSST is awesome for asteroids.

\section{LSST's potential for understanding asteroids}

MAF analysis of moving objects
completeness
characterization potential
how you can calculate this kind of stuff yourself


\section{Data challenges in discovering asteroids}

DM challenges (MOPS challenges overview)
Difference imaging will be okay
MOPS processing will be okay
ongoing work to make sure that it will all be okay

\section{Conclusion}

Timeline for LSST?
potential for better optimizing LSST for moving objects? 


\begin{thebibliography}{}

\bibitem[Amari \etal\ (1995)]{Amari_etal95}
{Amari, S., Hoppe, P., Zinner, E., \& Lewis R.S.} 1995,
\textit{Meteoritics}, 30, 490 

\end{thebibliography}

%\begin{discussion}
%\end{discussion}

\end{document}
