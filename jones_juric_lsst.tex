% iaus2esa.tex -- sample pages for Proceedings IAU Symposium document class
% (based on v1.0 cca2esam.tex)
% v1.04 released 17 May 2004 by TechBooks
%% small changes and additions made by KAvdH/IAU 4 June 2004
% Copyright (2004) International Astronomical Union

\NeedsTeXFormat{LaTeX2e}

\documentclass{iau}
\usepackage{graphicx}

\title[Asteroids in LSST] %% give here short title %%
{Asteroid Discovery and Characterization with the Large Synoptic Survey Telescope (LSST)}

\author[R. Lynne Jones \& Mario Juric]   %% give here short author list %%
{R. Lynne Jones$^1$
%%  \thanks{Present address: Department of Astronomy, University of
%%  Washington, Seattle, USA},
 \and Mario Juric$^2$}

\affiliation{$^1$University of Washington\\ email: {\tt lynnej@uw.edu} \\[\affilskip]
$^2$University of Washington \\email: {\tt mjuric@astro.astro.washington.edu}}

\pubyear{2015}
\volume{318}  %% insert here IAU Symposium No.
\setcounter{page}{1}
\jname{Asteroids: New Observations, New Models}
\editors{Steve Chesley, Alessandro Morbidelli, Robert Jedicke \&
  Davide Farnocchia eds.}
\begin{document}

\maketitle

\begin{abstract}
The Large Synoptic Survey Telescope (LSST) will be a ground-based,
optical, all-sky, rapid cadence survey project with tremendous
potential for discovering and characterizing asteroids.

With LSST's large 6.5m diameter primary mirror, a wide 9.6 square
degree field of view 3.2 Gigapixel camera, and rapid observational
cadence, LSST will discover more than 5 million asteroids over its ten
year survey lifetime. With a single visit limiting magnitude of 24.5
in $r$ band, LSST will be able to detect asteroids in the Main Belt
down to sub-kilometer sizes.  The current strawman for the LSST survey
strategy is to obtain two visits (each `visit' being a pair of
back-to-back 15s exposures) per field, separated by about 30 minutes,
covering the entire visible sky every 3-4 days throughout the
observing season, for ten years.

The catalogs generated by LSST will increase the known number of small
bodies in the Solar System by a factor of 10-100 times, among all
populations. The median number of observations for Main Belt asteroids
will be on the order of 200-300, with Near Earth Objects receiving a
median of 90 observations. These observations will be spread among
$ugrizy$ bandpasses, providing photometric colors and the allowing
sparse lightcurve inversion to determine rotation periods, spin axes, and shape information.

These catalogs will be created using automated detection software, the
LSST Moving Object Processing System (MOPS), that will take advantage
of the carefully characterized LSST optical system, cosmetically
clean camera, and recent improvements in difference imaging. Tests
with the prototype MOPS software indicate that linking detections (and thus
`discovery') will be possible at LSST depths with our working
model for the survey strategy, but evaluation of MOPS and improvements
in the survey strategy will continue. All data productsand software created by
LSST will be publicly available.
\keywords{Keyword1, keyword2, keyword3, etc.}
%% add here a maximum of 10 keywords, to be taken form the file <Keywords.txt>
\end{abstract}

\firstsection % if your document starts with a section,
                     % remove some space above using this command.
\section{Introduction}

The Large Synoptic Survey Telescope (LSST) is a next-generation survey
project created primarily to [science goals]
point to overview paper and science book

\section{LSST characteristics}
details about LSST's physical setup
mirror size
camera size
throughput estimates (from where)
limiting magnitude estimates (how arrived at)
refer to overview paper for updates

\section{LSST data products}
data products public! software open source!
data products that will be provided by lsst
level 1->sso detections + orbits,
  sso detections: position, shape, model fits (point source, trailed
  source, dipole)
level 2-> exquisite calibration, 
level 3-> user extensions, 
and trailing-> alerts
calibration requirements for lsst

\section{The LSST survey strategy}
lsst general survey strategy
not optimized for solar system objects, but obvious potential
working on plan to best balance science return among all goals
describe opsim and maf at a general level, as tools for simulations

\section{Survey strategy implications for moving objects}
extending capabilities of maf and simulation framework to moving
objects
calculate trailing losses and detection losses
include focal plane footprint
include m5 values from opsim and colors appropriate to sources
in particular (relevant for this symposium) we can look at PHA, NEO,
and MBA discovery and characterization
discovery rates with standard MOPS requirements/completeness
 various size distributions and effect on total number of objects
with extended MOPS requirements
with better MOPS requirements (and why we should continue to work on software)
how many objects can be detected as they trail
characterization - how many observations in general we obtain,
arclength
characterization example: detecting activity
characterization problem needing solution: determining colors
then colors + orbital elements -> families
and number of observations -> sparse lightcurve inversion
and colors + albedo relationship -> size distribution (better than
simple LF assumptions)
point users to repos (where?)

\section{Discovering moving objects with MOPS}
description of discovery requirements with current MOPS
requirements for MOPS to work:
 false positive rate (expected results, given DES/machine
 learning/etc) (driven harder here by other science areas)
   ccds within spec, crosstalk small, no optical ghosts, 
 capabilities of linking objects (results of previous experiments with
 prototypes)  improvements within MOPS
show number of chances to discovery an object and how many chances we
have as function of H 
ongoing work to understand limitations and capabilities of MOPS

\section{Conclusion}
timeline for lsst
briefly additional science cases
call for metrics and ways to make sure survey strategy is good for
moving objects



\begin{thebibliography}{}

\bibitem[Amari \etal\ (1995)]{Amari_etal95}
{Amari, S., Hoppe, P., Zinner, E., \& Lewis R.S.} 1995,
\textit{Meteoritics}, 30, 490 

\end{thebibliography}

%\begin{discussion}
%\end{discussion}

\end{document}
